\section{Controlled Diagnostic Protocol}

This section describes a controlled diagnostic protocol designed to examine the
qualitative behavior and internal consistency of the proposed framework under
progressive degradation of structural signals. The goal is not empirical
validation or performance benchmarking, but to clarify how diagnostic signals
emerge, stabilize, or dissolve as the underlying structure becomes weaker or
increasingly obscured by noise.

Real-world datasets rarely provide ground truth regarding structural
heterogeneity. Observed changes in predictive performance may arise from
transient noise, scale effects, or genuine structural mismatch, and these
factors are often entangled. To disentangle them conceptually, we consider
synthetic constructions in which specific sources of degradation can be
introduced in a controlled and interpretable manner.

From a Bayesian perspective, the protocol can be viewed as probing how posterior
support shifts between competing structural hypotheses --- a single global model
versus segmented alternatives --- as signal-to-noise conditions deteriorate.
Persistent support for the global hypothesis under increasing variability is
treated as a meaningful diagnostic outcome, rather than a failure of the method.

\subsection{Motivation}

The protocol is motivated by practical diagnostic questions rather than by
model optimization. In applied settings, practitioners often face ambiguity as
to whether observed instability reflects noise amplification or the emergence
of structurally distinct regimes. Standard first-line diagnostics are typically
insensitive to such distinctions.

The controlled constructions considered here are intended to clarify when the
proposed diagnostics indicate meaningful structural heterogeneity, when they
remain inconclusive, and when they explicitly favor retaining a global modeling
assumption.

\subsection{Protocol: \texorpdfstring{Time $\rightarrow$ Scale $\times$ Noise}{Time to Scale x Noise}}

Each dataset is organized into a sequence of discrete time slices. At the
initial time point, the data exhibit clear structural properties with minimal
noise and well-separated local behaviors. Subsequent slices apply controlled
transformations designed to progressively degrade structural identifiability.

Specifically, we consider two classes of transformations:
\begin{itemize}
    \item multiplicative scaling of the target variable, intended to simulate
    changes in exposure, volume, or magnitude without altering conditional
    structure;
    \item additive noise applied to covariates and/or targets, reducing the
    effective signal-to-noise ratio and blurring local distinctions.
\end{itemize}

These transformations are applied incrementally across time slices, producing a
trajectory from clearly structured regimes toward ambiguous or noise-dominated
settings. This design mirrors practical scenarios in which data quality or
comparability degrades gradually rather than abruptly.

\subsection{Expected Diagnostic Signals}

Rather than reporting performance metrics, the protocol examines qualitative and
structural diagnostic signals produced by the framework as conditions change.
The following aspects are monitored:
\begin{itemize}
    \item the stability or instability of the inferred number of zones across
    successive slices;
    \item the localization of high-energy boundaries in feature space and their
    persistence under increasing noise;
    \item relative changes in the adequacy of local predictive summaries compared
    to a global baseline;
    \item consistency of inferred segmentations across adjacent time slices.
\end{itemize}

Importantly, the absence of segmentation, or the collapse of previously detected
zones under increased noise, is treated as a valid and informative diagnostic
outcome. Such behavior indicates insufficient evidence to justify increased
structural complexity.

\subsection{Interpretation}

From a frequentist perspective, the protocol illustrates how regularized
structural diagnostics respond to decreasing signal strength and increasing
variability. Flat or weakly informative energy landscapes correspond to
noise-dominated regimes, while sharp transitions or stable elbows indicate
structural signals that persist despite degradation.

From a Bayesian perspective, these observations correspond to posterior mass
remaining concentrated on simpler structural hypotheses unless the data provide
consistent and localized evidence to support segmentation. In this sense, the
protocol highlights not only when segmentation is suggested, but also when it is
explicitly discouraged.

The purpose of this controlled diagnostic analysis is not empirical validation,
but to clarify how the proposed framework behaves under interpretable and
progressively degraded conditions. No claims are made regarding optimality,
predictive superiority, or comparative performance relative to alternative
methods. The constructions considered here serve solely as sanity checks for the
internal logic of the framework and as guidance for its practical diagnostic
interpretation.
