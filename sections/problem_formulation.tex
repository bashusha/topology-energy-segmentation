\section{Problem Formulation}

\subsection{Segmentation as a Diagnostic Hypothesis}

We consider a dataset consisting of covariates $X \in \mathbb{R}^d$ and target variables $F$, observed over discrete time periods. A global predictive model implicitly assumes that the conditional relationship between $X$ and $F$ is structurally homogeneous across the domain of interest.

Segmentation introduces an alternative hypothesis: that the data are better described by multiple regimes or zones, each governed by a locally coherent relationship between predictors and outcomes. Importantly, segmentation increases model complexity and operational burden, and therefore should be justified by sufficiently strong evidence.

In this work, segmentation is not treated as an optimization goal, but as a \emph{diagnostic outcome}: an indication that structural uncertainty in the model cannot be ignored.

From a Bayesian perspective, this diagnostic view can be interpreted as a comparison between competing structural hypotheses with different effective dimensionalities, where the absence of segmentation corresponds to a parsimonious prior preference for a single global model.

\subsection{Problem Setting and Inputs}

We consider observations $(X_i, F_i)$, where $X_i$ denotes a (possibly mixed-type) covariate vector and $F_i$ is a target variable of interest. The goal is to detect regions of the covariate space where a single global predictive relationship between $X$ and $F$ fails.

The output of the diagnostic procedure is a partition of selected landmark observations into a small number of zones, together with diagnostic quantities that explain and support the resulting segmentation. Importantly, the procedure may also return the trivial single-zone solution, indicating insufficient evidence to justify structural separation.
