\section{Conclusion}

We have introduced a topology-aware, energy-based diagnostic framework for
assessing structural uncertainty in predictive models.
The central idea of the framework is to treat segmentation not as a modeling
objective, but as a diagnostic hypothesis about the validity of a single global
predictive relationship.

By combining local geometric structure in the feature space with discrepancies
in target behavior, the proposed diagnostics identify regions where global
modeling assumptions systematically fail.
At the same time, the framework is deliberately conservative: in the absence of
stable structural signals, it favors simpler explanations and explicitly
returns the global model.

The energy-based formulation admits both frequentist and Bayesian
interpretations, providing a unified view of structural adequacy that is
compatible with established modeling traditions.
Importantly, these interpretations do not alter the computational procedure,
but offer complementary perspectives on the same diagnostic evidence.

Through controlled synthetic experiments, we illustrated how segmentation
emerges, stabilizes, or dissolves as data are progressively corrupted by scale
changes and noise.
These experiments reflect conditions commonly encountered in applied actuarial
and industrial settings, where low signal-to-noise ratios complicate structural
assessment.

Overall, the proposed framework is intended as a second-line diagnostic tool,
positioned between first-line drift monitoring and full model restructuring.
By formalizing when segmentation is warranted—and when it is not—the approach
supports more deliberate and interpretable responses to model instability in
complex, heterogeneous data.
