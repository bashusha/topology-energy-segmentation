\section{Synthetic Experiments}

\subsection{Motivation}

Real-world datasets rarely provide ground truth about structural heterogeneity.
To study the behavior of the proposed diagnostics under controlled conditions,
we construct synthetic datasets in which the degree of structural clarity is
systematically varied.

Rather than generating independent datasets, we adopt a
\emph{progressive corruption protocol}: starting from a clean, well-structured
dataset, we incrementally introduce scale distortions and noise.
This setup mirrors practical scenarios in which data quality degrades over time
due to changes in exposure, measurement error, or external conditions.

The goal of these experiments is not to optimize performance, but to examine
when segmentation emerges, stabilizes, or dissolves as structural signals become
weaker or obscured by noise.

\subsection{Protocol: \texorpdfstring{Time $\rightarrow$ Scale $\times$ Noise}{Time to Scale x Noise}}


Each dataset is organized into discrete time slices.
At the initial time point, the data exhibit clear structural properties with
minimal noise.
Subsequent slices apply controlled transformations designed to progressively
degrade structural identifiability.

Specifically, we consider:
\begin{itemize}
\item scaling of the target variable to simulate exposure growth or changing
magnitude;
\item additive noise applied to covariates and/or targets to reduce the
signal-to-noise ratio.
\end{itemize}

This design allows us to observe how diagnostic signals evolve as the data move
from clearly structured regimes toward ambiguity and noise-dominated settings.

\subsection{Evaluation criteria}

Diagnostic outcomes are assessed along several complementary dimensions:
\begin{itemize}
\item stability of the number of detected zones;
\item localization of high-energy boundaries in feature space;
\item improvement of local predictive models relative to a global baseline;
\item consistency of segmentation across time slices.
\end{itemize}

Importantly, the absence of segmentation is treated as a valid and informative
outcome, indicating insufficient evidence to justify increased structural
complexity.

From a Bayesian perspective, such outcomes correspond to posterior support
remaining concentrated on the global structural hypothesis despite increasing
data variability.
