\begin{abstract}

In many applied modeling settings, performance degradation is addressed by rebuilding or retuning predictive models in response to detected drift.
While effective in some cases, such interventions may obscure the underlying cause of instability, particularly when degradation arises from latent structural heterogeneity rather than gradual distributional change.
In this work, we propose a topology-aware, energy-based diagnostic framework for identifying when segmentation of a predictive model is warranted.

Throughout the paper, we use the term ``topology-aware'' in a loose but standard sense, referring to methods that explicitly respect neighborhood structure encoded by a graph, rather than to formal tools from topological data analysis such as persistent homology.

Our approach treats segmentation not as a primary modeling objective, but as a diagnostic hypothesis about structural uncertainty in the relationship between covariates and outcomes.
By combining local geometric structure in feature space with discrepancies in target behavior, we construct an energy functional whose minimization reveals stable zones of structural conflict.
The resulting segmentation highlights regions where a global model fails systematically, while avoiding overreaction to noise.

The framework admits both frequentist and Bayesian interpretations: as a regularized risk minimization problem with complexity penalties, or as an implicit evidence-based assessment of competing structural hypotheses.
Through controlled synthetic experiments, we illustrate how segmentation may emerge, stabilize, or dissolve as data are progressively corrupted by scale changes and noise, reflecting conditions commonly encountered in real-world actuarial and industrial applications.

\end{abstract}
