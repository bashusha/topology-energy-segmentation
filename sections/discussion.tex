\section{Discussion and Limitations}

The experimental results illustrate that topology-aware energy-based diagnostics
can distinguish between transient noise-induced fluctuations and persistent
structural heterogeneity.
In clean synthetic settings, segmentation emerges only where the
data-generating process contains genuine regime boundaries.
As noise increases, the framework naturally suppresses over-segmentation,
reflecting increased uncertainty rather than failure of the method.

A central aspect of the proposed framework is its conservative behavior.
Segmentation is not produced by default and is explicitly rejected when
diagnostic signals are weak, unstable, or inconsistent across validation
criteria.
Intentional abstention from segmentation is treated as a valid diagnostic
outcome, indicating insufficient evidence to justify increased model complexity.

\subsection{Role of the complexity penalty}

The complexity penalty parameter $\alpha$ plays a role analogous to model
selection penalties in classical statistical learning.
Rather than admitting a universally optimal choice, $\alpha$ is intended to be
used diagnostically.
Smaller values allow the method to reveal candidate structural inconsistencies,
while larger values enforce conservative behavior and favor global models.

In practice, we recommend sensitivity analysis over a narrow range of $\alpha$
values, combined with inspection of energy--complexity trade-offs (e.g.,
elbow-type behavior) and temporal stability when repeated data slices are
available.
This use of $\alpha$ aligns with the interpretation of segmentation as a
hypothesis to be evaluated rather than as an optimization target.

\subsection{Interpretational perspectives}

From a frequentist perspective, the framework can be viewed as a regularized
assessment of structural adequacy, balancing reductions in local inconsistency
against penalties for increased complexity.
From a Bayesian perspective, the same procedure corresponds to implicit
comparison of competing structural hypotheses under a prior favoring
parsimony.

These interpretations are complementary and lead to identical computational
procedures.
The Bayesian framing is intended as an interpretational aid rather than as a
fully specified probabilistic generative model.

\subsection{Computational considerations}

The framework involves several computational trade-offs.
In particular, the refinement phase based on rich distributional disagreement
metrics can be computationally expensive.
For this reason, it is applied only to a restricted set of candidate
segmentations identified during the surrogate phase.

The use of landmarks and sparse locality graphs further reflects a deliberate
design choice to balance diagnostic sensitivity with computational feasibility.
The framework is intended primarily for offline structural assessment rather
than for real-time deployment.

\subsection{Limitations}

The proposed diagnostics are subject to several important limitations.
Reliable detection of structural heterogeneity requires sufficient data within
each candidate zone; rare regimes with very small sample sizes may remain
undetectable.
The method also depends on the construction of a meaningful locality graph; poor
choices of distance metrics or neighborhood sizes can degrade diagnostic quality.

Finally, while the framework is designed for low signal-to-noise tabular data, it
does not guarantee segmentation in all such settings.
In highly noisy regimes, the diagnostics correctly favor simpler explanations,
but this may limit interpretability in cases where weak structure is present but
statistically indistinguishable from noise.

\subsection{Extensions and future directions}

Several extensions of the framework are conceptually straightforward but are not
systematically explored in this work.
These include the use of more expressive topological diagnostics, bootstrap-based
stability assessments, and richer Bayesian formulations with explicit priors over
segmentations.

We leave a comprehensive empirical evaluation of these extensions for future
work.
