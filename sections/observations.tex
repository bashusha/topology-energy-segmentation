\section{Experimental Observations}

\subsection{Emergence of structural segmentation}

In clean synthetic settings with explicit structural heterogeneity, the
energy-based diagnostics consistently identify a small number of stable zones.
These zones correspond to regions in which local target behavior cannot be
reconciled by a single global predictive model.

The emergence of segmentation is characterized by pronounced changes in the
energy trajectory as a function of the number of zones.
In particular, further merging beyond certain points leads to
disproportionate increases in boundary disagreement, indicating the presence of
structurally meaningful divisions.

\subsection{Effect of scale transformations}

Pure scale transformations of the target variable, applied uniformly across the
dataset, do not induce artificial segmentation.
Although absolute discrepancies in the target increase, the relative structure
of local neighborhoods remains coherent.

As a result, the energy landscape does not exhibit stable minima at higher
numbers of zones, and the diagnostics correctly favor the global model.
This behavior illustrates that the framework responds to structural
inconsistency rather than to magnitude effects alone.

\subsection{Impact of noise on diagnostic stability}

As noise is progressively added to covariates and targets, the diagnostics
naturally suppress segmentation.
Energy differences flatten, and the complexity penalty dominates, favoring
simpler explanations.

The absence of segmentation in these regimes reflects increased uncertainty
rather than failure of the method.
From a Bayesian perspective, this corresponds to posterior mass concentrating on
simpler structural hypotheses as the likelihood becomes less informative.

\subsection{Comparison with geometry-only methods}

Geometry-driven clustering methods respond primarily to changes in the structure
of the feature space.
In contrast, the proposed diagnostics detect conflicts driven by target
inconsistency even when geometric separation is weak or absent.

This contrast highlights the role of the dual-frame construction:
segmentation is triggered by disagreement in $F \mid X$, not by geometry alone.

\subsection{Temporal consistency}

When experiments are organized into time slices with gradually increasing
degradation, stable segmentation patterns persist only while structural
heterogeneity remains detectable.

Transient zones that appear only under increased noise or scaling lack temporal
consistency and are not reinforced by energy-based criteria.
Temporal persistence therefore serves as an additional validation signal,
supporting segmentation only when it reflects stable structural obstacles.
